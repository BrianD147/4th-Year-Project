%!TEX root = project.tex

\chapter{Technology Review}
About seven to ten pages.
\begin{itemize}
\item Describe each of the technologies you used at a conceptual level. Standards, Database Model (e.g. MongoDB, CouchDB), XMl, WSDL, JSON, JAXP.
\item Use references (IEEE format, e.g. [1]), Books, Papers, URLs (timestamp) – sources should be authoritative. 
\end{itemize}

\section{Machine Learning}
Machine learning (ML) is an application of artificial intelligence (AI) that provides systems the ability to automatically learn and improve from experience without being explicitly programmed. Machine learning focuses on the development of computer programs that can access data and use it learn for themselves.

One of the current trends in modern day software development is the exploration into the world of Artificial intelligence.
With the recent boom of Artificial Intelligence and Machine learning, we thought it would be a good idea to jump onto the bandwagon and see what it was all about.
In this paper we will describe in detail our process of creating an unity environment with the goal of teaching an agent how to play soccer, the challenges encountered throughout and the research done into the areas of Machine Learning and the Unity environment.  We will then discuss how we implemented this idea with ideas acquired from our research. We will then discuss our results, what we learned and how we could possible improve on in the future with further research.

\section{Neural Networks}
Neural network, a computer program that operates in a manner inspired by the natural neural network in the brain. The objective of such artificial neural networks is to perform such cognitive functions as problem solving and machine learning.
\section{Unity}
Unity is a cross-platform real-time engine developed by Unity Technologies. The engine can be used to create both three-dimensional and two-dimensional games as well as simulations for its many platforms.
\section{Python}
Python is an interpreted, object-oriented, high-level programming language with dynamic semantics. Its high-level built in data structures, combined with dynamic typing and dynamic binding, make it very attractive for Rapid Application Development, as well as for use as a scripting or glue language to connect existing components together. Python's simple, easy to learn syntax emphasizes readability and therefore reduces the cost of program maintenance. 


\section{XML}
Here's some nicely formatted XML:
\begin{minted}{xml}
<this>
  <looks lookswhat="good">
    Good
  </looks>
</this>
\end{minted}