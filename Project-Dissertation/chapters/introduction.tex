%!TEX root = project.tex

\chapter{Introduction}

\section{Overview}
This chapter will outline the context and scope of the project. It will explain the objectives we had for the project and the specifications for the software, as well as detail our initial ideas and thoughts behind the design and it's implementation. 

\section{Idea}
At the beginning of our 4th year in Software Development, we were tasked with coming up with an idea that would be suited to a level 8 final year project, so we knew we would have to design something that combined multiple different technologies and languages, and also challenged our own knowledge thus far. Our decision on what we were going to design sparked into life while we were exploring different possibilities of what to develop and what platform and languages we could work with. We stumbled upon a video from an event called the "Robocup", where teams of Artificially Intelligent Robots would play a game of soccer. After seeing this and being inspired, we eventually settled on the idea to develop some form of AI controlled agent. We had no experience with physical hardware like the Robocup utilised, and we wanted to find an environment that would suit what we aimed to do. For this reason, we decided to used the Unity game engine environment to act as the base for our AI controlled agents. We had the option to used Unity in a previous years module, however we had never really delved into the environment in much detail, and thus we still had a lot to learn about Unity's inner workings and plugins.

In order to control our AI agents, we would need to design a neural network somehow. We knew we were going to have a module in AI during our second semester, however at the time we had never touched anything to do with AI or neural networks, and our research into the field seemed to point us towards using python. Python seemed to be perfect to design the machine learning side of our project, as the language has a vast array of open source libraries free for us to use.

In the end we finished up our research and design with an idea to create a 3D Unity environment of a football stadium, with AI controlled agents playing football logically, all controlled by a python neural network externally to the Unity environment.


\section{The Application}
This application is a simulation of agents moving independently in a 3D through a neural network developed using python. We plan to have these agents recognize the object around them in the environment and use these to their advantage. With this we hope to add some personalities to each of these agents so that they make decisions based on what position they are in, and what to prioritize when in certain situations. We also hope to implement a win condition so that the simulation stops after a certain amount of time has passed or a certain amount of goals have been scored by a single team or agent.

\section{Scope}
As this is a final year software development project, we felt that this idea would provide many learning outcomes associated with the scope at this level. We were encouraged to use technologies that were new to us so with little to no experience in both unity and python we felt our idea slotted right into this category. As the project developed further we dove into other possible technologies that would benefit our development and working environment.
%end - Ryan Conway

The purpose of this application is to train and reinforce the actions of particular agents on opposing teams, so that they work together to achieve an objective. In this case that objective is to score goals until a determined limit is reached. The scope of this project incorporates many technologies, such as Unity, ML-Agents Toolkit, Jupyter Notebooks and Python, et al. 

This project intends to show the applications of reinforced learning algorithms in a neural network, how they are applied within and the utility of such learning algorithms in more mundane day to day applications.



The Objectives of the project can be described as such:

\begin{itemize}
  \item Create Unity environment for testing, complete with Agent, Ball, Goal and a rudimentary form of player movement through user input (Just to make sure 
everything's chooching along nicely)
  \item Integration of ML Agents 
  \item Integration of Jupyter Notebooks
  \item Testing cross functionality of technologies through random input
  \item Pulling observations from environment to Jupyter Notebooks
  \item Passing through training parameters to Agents
  \item Reinforced learning through positive/negative rewarding
  \item Decreasing the time taken to train a singular or multiple agents
  \item Creating a team dynamic (or the illusion of)
\end{itemize}