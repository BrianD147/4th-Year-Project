%!TEX root = project.tex

\chapter{Introduction}

% May need another look at

\section{Idea}
At the beginning of our 4th year in Software Development, we were tasked with coming up with a project idea that would challenge our knowledge. ------
The idea came about when we were exploring different possibilities on what to develop, what platform to develop it on and what languages could we explore. We were encouraged to develop this on a platform or language we were not used to, so we tried to aim for something that was new to us. We stumbled upon a video from an event called the “Robocup”, where teams of Artificially Intelligent Robots would play a game of soccer. After seeing this, we eventually settled on the idea that would develop some form of AI controlled agent. We had no experience in the hardware side of development, we went to find an environment that would best suit what we wanted, and that when we decided on using Unity.


We decided on using both Unity and Python for this project as the environment Unity provided made it the perfect playground to design our agents and the stadium, and the use of python for the Machine learning due to its vast amount of open source libraries available to use.

With our technologies picked we set out at list of goals we wanted this project to achieve. 
\begin{itemize}
  \item Create a working environment in unity suitable to our desires
  \item Find a way to have Unity and Python communicate
  \item Get a single agent moving in the unity environment through python
  \item Train an agent using a neural network
  \item Have more than one agent moving in the work environment
  \item Have designated positions for each agent
\end{itemize}

\section{The Application}
This application is a simulation of agents moving independently in a 3D through a neural network developed using python. We plan to have these agents recognize the object around them in the environment and use these to their advantage. With this we hope to add some personalities to each of these agents so that they make decisions based on what position they are in, and what to prioritize when in certain situations. We also hope to implement a win condition so that the simulation stops after a certain amount of time has passed or a certain amount of goals has been scored by a single team or agent.

\section{Scope}
As this is a final year software development project, we felt that this idea would provide many learning outcomes associated with the scope at this level. We were encouraged to use technologies that were new to us so with little to no experience in both unity and python we felt our idea slotted right into this category. As the project developed further we dove into other possible technologies that would benefit our development and working environment.
%end - Ryan Conway




The introduction should be about three to five pages long.
Make sure you use references~\cite{einstein}

The purpose of this application is to train and reinforce the actions of particular agents on opposing teams, so that they work together to achieve an objective. In this case that objective is to score goals until a determined limit is reached. The scope of this project incorporates many technologies, such as Unity, ML-Agents Toolkit, Jupyter Notebooks and Python, et al. 

This project intends to show the applications of reinforced learning algorithms in a neural network, how they are applied within and the utility of such learning algorithms in more mundane day to day applications.



The Objectives of the project can be described as such:

\begin{itemize}
  \item Create Unity environment for testing, complete with Agent, Ball, Goal and a rudimentary form of player movement through user input (Just to make sure 
everything's chooching along nicely)
  \item Integration of ML Agents 
  \item Integration of Jupyter Notebooks
  \item Testing cross functionality of technologies through random input
  \item Pulling observations from environment to Jupyter Notebooks
  \item Passing through training parameters to Agents
  \item Reinforced learning through positive/negative rewarding
  \item Decreasing the time taken to train a singular or multiple agents
  \item Creating a team dynamic (or the illusion of)
\end{itemize}