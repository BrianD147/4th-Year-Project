%!TEX root = project.tex

\chapter{Introduction}

% May need another look at

Idea:
At the beginning of our 4th year in Software Development, we were tasked with coming up with a project idea that would challenge our knowledge. ------
The idea came about when we were exploring different possibilities on what to develop, what platform to develop it on and what languages could we explore. We were encouraged to develop this on a platform or language we were not used to, so we tried to aim for something that was new to us. We stumbled upon a video from an event called the “Robocup”, where teams of Artificially Intelligent Robots would play a game of soccer. After seeing this, we eventually settled on the idea that would develop some form of AI controlled agent. We had no experience in the hardware side of development, we went to find an environment that would best suit what we wanted, and that when we decided on using Unity.


We decided on using both Unity and Python for this project as the environment Unity provided made it the perfect playground to design our agents and the stadium, and the use of python for the Machine learning due to its vast amount of open source libraries available to use.

With our technologies picked we set out at list of goals we wanted this project to achieve. 
\begin{itemize}
  \item Create a working environment in unity suitable to our desires
  \item Find a way to have Unity and Python communicate
  \item Get a single agent moving in the unity environment through python
  \item Train an agent using a neural network
  \item Have more than one agent moving in the work environment
  \item Have designated positions for each agent
\end{itemize}
%end




Machine learning (ML) is an application of artificial intelligence (AI) that provides systems the ability to automatically learn and improve from experience without being explicitly programmed. Machine learning focuses on the development of computer programs that can access data and use it learn for themselves.

One of the current trends in modern day software development is the exploration into the world of Artificial intelligence.
With the recent boom of Artificial Intelligence and Machine learning, we thought it would be a good idea to jump onto the bandwagon and see what it was all about.
In this paper we will describe in detail our process of creating an unity environment with the goal of teaching an agent how to play soccer, the challenges encountered throughout and the research done into the areas of Machine Learning and the Unity environment.  We will then discuss how we implemented this idea with ideas acquired from our research. We will then discuss our results, what we learned and how we could possible improve on in the future with further research.

\section{Neural Networks}
Neural network, a computer program that operates in a manner inspired by the natural neural network in the brain. The objective of such artificial neural networks is to perform such cognitive functions as problem solving and machine learning.
\section{Unity}
Unity is a cross-platform real-time engine developed by Unity Technologies. The engine can be used to create both three-dimensional and two-dimensional games as well as simulations for its many platforms.
\section{Python}
Python is an interpreted, object-oriented, high-level programming language with dynamic semantics. Its high-level built in data structures, combined with dynamic typing and dynamic binding, make it very attractive for Rapid Application Development, as well as for use as a scripting or glue language to connect existing components together. Python's simple, easy to learn syntax emphasizes readability and therefore reduces the cost of program maintenance. 



The introduction should be about three to five pages long.
Make sure you use references~\cite{einstein}

The purpose of this application is to train and reinforce the actions of particular agents on opposing teams, so that they work together to achieve an objective. In this case that objective is to score goals until a determined limit is reached. The scope of this project incorporates many technologies, such as Unity, ML-Agents Toolkit, Jupyter Notebooks and Python, et al. 

This project intends to show the applications of reinforced learning algorithms in a neural network, how they are applied within and the utility of such learning algorithms in more mundane day to day applications.



The Objectives of the project can be described as such:

\begin{itemize}
  \item Create Unity environment for testing, complete with Agent, Ball, Goal and a rudimentary form of player movement through user input (Just to make sure 
everything's chooching along nicely)
  \item Integration of ML Agents 
  \item Integration of Jupyter Notebooks
  \item Testing cross functionality of technologies through random input
  \item Pulling observations from environment to Jupyter Notebooks
  \item Passing through training parameters to Agents
  \item Reinforced learning through positive/negative rewarding
  \item Decreasing the time taken to train a singular or multiple agents
  \item Creating a team dynamic (or the illusion of)
\end{itemize}