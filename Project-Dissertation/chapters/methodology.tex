%!TEX root = project.tex

\chapter{Methodology}
About one to two pages.
Describe the way you went about your project:
\begin{itemize}
\item Agile / incremental and iterative approach to development. Planning, meetings.
\item What about validation and testing? Junit or some other framework.
\item If team based, did you use GitHub during the development process.
\item Selection criteria for algorithms, languages, platforms and technologies.
\end{itemize}

In this section we will review our approach to the project development, describe the methodology used and how it was implemented and how the development of the project progressed overall. At the conclusion of this methodology the reader should be left with a good sense of the project scale, as well as the steps taken throughout.

For this project we wanted to use an iterative design process, so changes and issues could easily be kept track of and dealt with in an easy and efficient manner. As such we felt the Agile method to be the best fit for us, as one of the team members has had previous experience in industry during a summer internship. His first hand experience helped guide us throughout the project, and gave the rest of us a feel on how it may be like to work in industry in the future.

(Definitions? Scrum master, stand up, etc)

The Agile method is a particular approach to project management, which is ubiquitous in the environment of Software Development. This method assists teams in responding to the unpredictability of constructing software, such as bugs in code, changing requirements from the client, etc. Agile uses incremental/iterative work sequences commonly known as Sprints, which is a period of time allocated for a particular phase of a project. Time is a precious resource when developing software, and as such a Sprint is considered complete once the time period expires. While some objectives set out for the Sprint may not have been realised, these additions will need to be put on the back burner in order to keep up with the Agile development cycle.

Some general principles of Agile include:

\begin{itemize}
    \item{Satisfy the client and continually develop software}
    \item{Changes in requirements are embraced}
    \item{Frequent delivery of working software} 
    \item{Close working relationship between client and developer(s)}
    \item{Face to face communication is integral to cohesive development, and working relations as a whole}
\end{itemize}

\section{Sprint 1: Idea}
	(Spent time brainstorming ideas, technologies, platforms, hardware, etc)
	
	Our first sprint was used to spitball potential ideas for a project, including what problem could we tackle, relevant technologies we could apply to said problem and in what manner we could approach it. 
	
\section{Sprint 2: Research}
	Spent researching our topic, looking into -relevant- technologies and languages 
	
	The next Sprint focused on researching our materials

\section{Sprint 3: Development 1}
	Set up environment, Github, etc

\section{Sprint 4: Development 2}
	Integration testing of ML-Agents, Unity, Notebooks, etc

\section{Sprint 5: Development 3}

\section{Sprint 6: Development 4}
	Final testing, padding out functionality

\section{Sprint 7: Cleanup \& Dissertation} 
	Writing up dissertation



\begin{figure}
  \centering
  \begin{tikzpicture}[node distance=6cm]
  \node (a) [rect] {A Big Blue Block};
  \node (b) [oval, right of=a] {And His Oval Friend};
  \draw [line] (a) -- (b);
  \end{tikzpicture}
  \caption{Nice pictures}
  \label{tikz:graphs}
\end{figure}