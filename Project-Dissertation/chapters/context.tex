%!TEX root = project.tex

\section{Objectives}
The objectives of this project are:

\begin{itemize}
    \item{Design a working 3D environment in Unity to best suit our needs.}
    \item{Find a way to connect Unity and python together, allowing for data to be transferred between both the environment and the external code.}
    \item{Pass movements into the Unity environment using the external code.}
    \item{Pull observations from the Unity environment back into the external code.}
    \item{Control the agents using the external neural network.}
\end{itemize}

\section{Chapter Summaries}

\subsection{Introduction}
This chapter contains the context for the entire project covering where the idea came from, what the project is about, what the objectives are for the project going forward, and the location and different elements of our GitHub Repository.

\subsection{Methodology}
This chapter describes the way the project was approached and managed. It also gives a description of how the project was researched and developed.

\subsection{Technology}
This chapter discussed the technologies that were researched and used in our project. It gives an insight into why the technologies used were chosen and what alternatives could have been used.

\subsection{System Design}
This chapter gives an explanation of how the entire system architecture was designed and how it all connects together. Diagrams are provided to further explain each individual element of the system.

\subsection{System Evaluation}
This chapter evaluates the project and the progress that was made towards it's completion, and highlights where the project could have been improved upon. It discussed the problems faced and the impact these had on the development of the project.

\subsection{Conclusion}
The conclusion gives a summary of our findings, outcomes and experiences we all had during the development of this project.

\section{GitHub Repository}
The GitHub repository for this project can be found at:

    https://github.com/BrianD147/4th-Year-Project. 

The sections below describe the different components in the repository and a link to each part

\subsection{README}
This contains a brief introduction and description of each component of the system. It also contains an installation guide as to how to install and run the project.

\subsection{4th-Year-Project}
This contains the main Unity environment code used to design the 3d stadium environment and the agents. Also contains the builds necessary to work with the notebooks within this project.

\subsection{Project Dissertation}
This contains a latex project used to write and develop out project dissertation and a PDF of the complete dissertation.


\subsection{ML-Agents}
This contains the ml agents python package, which is part of the ML-Agents Toolkit. This is a Python API that allows direct interaction with the Unity game engine as well as a collection of trainers and algorithms to train agents in the Unity environment.


\subsection{Notebooks}
This contains Jupyter notebooks used to run the python scripts and connect them to the Unity environment, with help from the ml-agents Python API. This folder contains 4 notebooks that each run different simulation types, and another notebook with an example of a Keras Neural Network. 


\subsection{Group Project 2018}
This contains an initial document stating our initial goal and a road map we wished to adhere by as development progressed.

\subsection{IronPython VS ML-Agents}
This contains a document comparing the IronPython plugin with ML-Agents and discussing the advantages ML-Agents has over IronPython, concluding in our reasons for choosing ML-Agents going forward.

\subsection{Input Data}
This contains a file stating the input data and attributes intended to be passed into the neural network, and the set of outputs the network would output. Along with template data arrays for inputs and outputs.


\subsection{Setup for Player Controller}
This contains a brief explanation of how user controlled movement can be incorporated into the ML-Agents Academy and Brain prefabs.

