%!TEX root = project.tex

\chapter{Conclusion}
About three pages.

\section{Conclusion}
This chapter will summarise the project in terms of the objectives of the initial proposal and will discuss the findings and outcomes of the project now it has concluded.

This project proposed the development of a 3d environment build in Unity consisting of multiple AI controlled agents playing football. External python code was proposed to run the neural network controlling the agents external to the Unity environment. 

\section{Goals and Objectives}

The following shows our original objectives for the project, whether we managed to fulfill said objective, and a short comment on each objective, either how we managed to succeed or why we failed.

\begin{itemize}
    \item{Design a working 3d environment in Unity to best suit our needs
[PASS]
Our aim to design a working 3D Unity environment to hold our agents in was successful, with a football pitch surrounded by clear walls working perfectly to house the AI agents and ball.}

    \item{Find a way to connect Unity and python together, allowing for data to be transferred between both the environment and the external code
[PASS]
We managed to connect the Unity environment to external python scripts using Jupyter Notebooks, acting as a buffer in between both and providing a clean and easy to understand interface for running the code.}

    \item{Pass movements into the Unity environment using the external code
[PASS]
We successfully passed random actions from the Jupyter notebooks into the Unity environment, moving the agents to random positions on the field on each frame.}

    \item{Pull observations from the Unity environment back into the external code
[FAIL]
We failed to find a way of passing any observations from the Unity environment back into the Jupyter notebooks, and thus into any python scripts.}

    \item{Control the agents using the external neural network
[FAIL]
We were unable to control any agents from an external neural network due to the previous objective failure. Having no observations to supply as input to any neural network meant controlling the agents wasn't possible.}
\end{itemize}

\section{Retrospective of the project}
The following are the findings and outcomes from this project:
\begin{itemize}
    \item{A better decision could have been made in terms of using ML-Agents, since the plugin is actually still in early beta.}
    \item{ML-Agents went from version 0.5 to version 0.6 in the midst of our development, and with it many aspects of how the plugin worked previously changed, this meant that all the documentation on the previous version was changed also.}
    \item{The documentation for ML-Agents wasn't actually as helpful as we originally thought, lots of pre-built examples exist, however an actual breakdown of what went into these examples does not.}
    \item{For any future development we may be involved in, versioning has clearly stated itself as being of utmost importance, for example, ML-Agents version 0.5 will only work with python 3, something that we were never made aware of in the ML-Agents documentation (ML-Agents 0.5 will not run with the latest version of ML-Agents, meaning ML-Agents 0.6 will not run any project developed in 0.5)}
    \item{A GPU was necessary to run the training notebooks, and hardware became an issue with only one of us having a laptop capable of running the intense training simulations through the notebooks.}
    \item{Hardware was also an issue running the Unity environment after intense training simulations had taken place.}
    \item{Scrum Agile Methodology provided good structure to our development process, however mid way through development, with so many setbacks and subsequent blocking issues, the sprint schedule was effected in the later stages, messing with our overall time management.}
    \item{GitHub has a limit to the size of file that can be uploaded, meaning that many of the ml-agents files could not be uploaded directly, as they exceeded the 100MB allowance}
\end{itemize}


The whole development of the project was a challenging and rewarding experience. Although things didn't work out as we initially intended, and many aspects of the development cycle were rather challenging with a lack of useful documentation to work with, we all gained a great insight into the technologies used, as well as a greater bond supporting each other through the many struggles and hardships, whilst managing to keep a positive outlook throughout the development. Overall we are proud with the research and development we managed to fulfill, and we all have learned a lot despite the project not coming together as we had initially hoped.


\begin{itemize}
\item Briefly summarise your context and objectives (a few lines).
\item Highlight your findings from the evaluation section / chapter and any opportunities identified.
\end{itemize}